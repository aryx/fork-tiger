\documentclass[twoside]{report}
\usepackage{epsfig}
\usepackage{noweb}
%\noweboptions{breakcode}

\title{Tiger frontend compiler for {\tt C--}}
\author{Paul Govereau}
\date{\today}
\begin{document}
\maketitle
\ifhtml\else\tableofcontents\fi
\clearpage

\chapter{Introduction}
This document describes the source code for a Tiger language front end
to the \nwanchorto{http://www.cminusminus.org}{Quick {\tt C--}
compiler}. The Tiger language is described in "Modern Compiler
Implementation in ML"\cite{appel}. To quote from Appendix A of this
book:
\begin{quote}
The Tiger language is a small language with nested functions, record
values with implicit pointers, arrays, integer and string variables,
and a few simple structured control constructs.
\end{quote}
In addition, we have added exceptions to the basic Tiger language.
Full source code and documentation can be found at the {\tt C--}
website, or in the Quick {\tt C--} CVS repository under {\tt
frontends/tiger}.

% --------------------------------------------------------------------
\section{Guide to the Source Code}
% --------------------------------------------------------------------
The sources are dived into several modules, each of which is described
below. Figure~\ref{fig:tiger} shows the dependencies between the major
modules in the compiler.
%
\begin{figure}
\epsfig{file=tiger.eps}
\caption{Module Dependencies}
\label{fig:tiger}
\end{figure}

% --------------------------------------------------------------------
\paragraph{Abstract Syntax}
The abstract syntax for Tiger programs is described in the {\tt Ast}
module. Symbols and tables of symbols are implemented in a companion
module {\tt Symbol}. As usual, Tiger source programs are converted
into abstract syntax by the parser. The abstract syntax is then used
by the rest of the compiler as the representation of programs.

\bigskip
\begin{tabular}{ll}
\ifhtml\nwanchorto{symbol.html}{symbol.nw}\else symbol.nw\fi & 
 Symbols and symbol tables.\\
\ifhtml\nwanchorto{ast.html}{ast.nw}\else ast.nw\fi & 
 Abstract Syntax Trees.\\
\end{tabular}

% --------------------------------------------------------------------
\paragraph{Analysis}
The analysis phase verifies that the abstract syntax tree represents a
semanticly correct program. This process includes type checking and
verifying that expressions are used in a proper context (e.g. a
{\tt break} can only occur within a loop). The semantic analysis uses an
environment to keep track of type definitions, variable bindings and
other information. The environment is implemented in the
{\tt Environment} module, and the analysis is implemented in the
{\tt Semantics} module.

\bigskip
\begin{tabular}{ll}
\ifhtml\nwanchorto{environment.html}{environment.nw}\else environment.nw\fi & 
 Environments. \\
\ifhtml\nwanchorto{semantics.html}{semantics.nw}\else semantics.nw\fi & 
 Semantic analysis. \\
\end{tabular}

% --------------------------------------------------------------------
\paragraph{Translation}
The {\tt Translate} module translates an abstract syntax tree into the
intermediate representation used by the rest of the compiler. The
intermediate representation is defined in the {\tt Tree} module.

\bigskip
\begin{tabular}{ll}
\ifhtml\nwanchorto{tree.html}{tree.nw}\else tree.nw\fi & 
 The intermediate representation.\\
\ifhtml\nwanchorto{translate.html}{translate.nw}\else translate.nw\fi & 
 Translates ASTs to intermediate representation.\\
\end{tabular}

% --------------------------------------------------------------------
\paragraph{Code Generation}
The code generator translates the intermediate representation into a
{\tt C--} program. Each Tiger function is translated into a {\tt C--}
function by the Tiger compiler. The stack frame for each function is
managed by the {\tt Frame} module. The {\tt Codegen} module outputs
{\tt C--} programs from the intermediate representation and
information contained in the {\tt Frame} module.

\bigskip
\begin{tabular}{ll}
\ifhtml\nwanchorto{frame.html}{frame.nw}\else frame.nw\fi & 
 Stack frames.\\
\ifhtml\nwanchorto{codegen.html}{codegen.nw}\else codegen.nw\fi & 
 Generates {\tt C--} code.\\
\end{tabular}

% --------------------------------------------------------------------
\paragraph{Tiger Runtime System}
Compiled tiger programs must be linked with the tiger runtime system
to create a complete program. The runtime system contains the startup
code, a simple copying garbage collector, and a library of standard
functions.

\bigskip
\begin{tabular}{ll}
\ifhtml\nwanchorto{gc.html}{gc.nw}\else gc.nw\fi & 
 Garbage collector.\\
\ifhtml\nwanchorto{stdlib.html}{stdlib.nw}\else stdlib.nw\fi & 
 Tiger standard library functions.\\
\ifhtml\nwanchorto{runtime.html}{runtime.nw}\else runtime.nw\fi & 
 Startup code.\\
\end{tabular}

% --------------------------------------------------------------------
\paragraph{Other Modules}
The Tiger compiler also contains several support modules. These
modules are listed below---the complete sources can be found in the
appendices.

\bigskip
\begin{tabular}{ll}
\ifhtml\nwanchorto{driver.html}{driver.nw}\else driver.nw\fi & 
 The compiler driver.\\
\ifhtml\nwanchorto{error.html}{error.nw}\else error.nw\fi & 
 Error reporting.\\
\ifhtml\nwanchorto{option.html}{option.nw}\else option.nw\fi & 
 Command line options. \\
\ifhtml\nwanchorto{parser.html}{parser.nw}\else parser.nw\fi & 
 Parser and scanner.\\
\ifhtml\nwanchorto{canonical.html}{canonical.nw}\else canonical.nw\fi & 
 Linearizes expression trees.\\
\end{tabular}

% --------------------------------------------------------------------
% include documentation for modules in ps/pdf files
% --------------------------------------------------------------------
\ifhtml\else
\chapter{Abstract Syntax of Tiger}
\input{symbol.inc}
\input{ast.inc}

\chapter{Analysis}
\input{environment.inc}
\input{semantics.inc}

\chapter{Translation}
\input{tree.inc}
\input{translate.inc}

\chapter{Code Generation}
\input{frame.inc}
\input{codegen.inc}

\chapter{Tiger Runtime System}
\input{gc.inc}
\input{stdlib.inc}
\input{runtime.inc}

\begin{thebibliography}{27}
\bibitem[1]{appel}
Andrew~W. Appel.
\newblock {\em Modern compiler implementation in ML: basic techniques}.
\newblock Cambridge University Press, 1998.
\newblock ISBN 0-521-58775-1.
\bibitem[2]{cmm1}
Norman Ramsey and Simon~L. Peyton~Jones.
\newblock A single intermediate language that supports multiple implementations
  of exceptions.
\newblock {\em Proceedings of the ACM SIGPLAN~'00 Conference on Programming
  Language Design and Implementation, {\em in} SIGPLAN Notices}, 35\penalty0
  (5):\penalty0 285--298, May 2000.
\end{thebibliography}

\appendix
\chapter{Driver and Utilities}
\input{driver.inc}
\input{error.inc}
\input{option.inc}

\chapter{Scanner and Parser}
\input{parser.inc}

\chapter{Linearize Algorithm}
\input{canonical.inc}

\fi
\end{document}
